%%%%%%%%%%%%%%%%%%%
%-----Paquets-----%
%%%%%%%%%%%%%%%%%%%

\usepackage[utf8]{inputenc}
\usepackage[french]{babel}
\usepackage{listing}
\usepackage{amsmath, amsfonts, amssymb, amsthm}
\usepackage{graphicx}
\usepackage{geometry}
\usepackage{xcolor}
\usepackage{pagecolor}
\usepackage[most,skins,breakable]{tcolorbox}
\usepackage{extsizes} % Permet d'utiliser les tailles de 17 et 20 pt
\usepackage{fancyhdr} % Faire des entêtes
\usepackage{lastpage} % Avoir le nombre de page avec \pageref{Lastpage}
\usepackage{soulutf8} % Pour surligner avec \hlc
\usepackage{tabularx} % Pour de meilleurs tableaux
\usepackage{pifont} % Pour utiliser des petits signes mignons
\usepackage[export]{adjustbox} % \includegraphics[width=0.5\textwidth, right] 
\usepackage{cellspace} 
\usepackage[inline]{enumitem} % enumerate* pour énumérer en ligne
\usepackage{setspace} % \doublespacing \onehalfspacing
\usepackage{calligra} % Pour \mathcal correctement les C
\usepackage{calrsfs} % Idem
\usepackage{multicol} % Colonnes
\usepackage{hyperref} % Lien hypertext
\usepackage{titling} % Remonter le titre
\usepackage{pdfpages}
\usepackage{titlesec}
\usepackage{multirow}
\usepackage{pstricks-add}
\usepackage{fp}
\usepackage{xlop}
\usepackage{svg}


\usepackage{cleveref} % Référence aux thm 
\setlength{\columnseprule}{0.5pt}
\usepackage{catchfilebetweentags} % inclure juste des parties de fichier entree %<*tag> et %</tag>


%%%%%%%%%%%%%%%%
%-----Tikz-----%
%%%%%%%%%%%%%%%%
\usepackage{tkz-tab} % Tableau de variations*
\usepackage{pgfplots}
\usepackage{tikz}
\usetikzlibrary{shapes.misc}

\tikzset{cross/.style={cross out, draw=black, minimum size=2*(#1-\pgflinewidth), inner sep=0pt, outer sep=0pt},
	%default radius will be 5pt. 
	cross/.default={5pt}}


%%%%%%%%%%%%%%%%%%%%
%-----Théorème-----%
%%%%%%%%%%%%%%%%%%%%

\theoremstyle{definition}
\newtheorem{thm}{\r Théorème}
\newtheorem{prop}{\r Propriété}
\newtheorem{propal}{\r Proposition}
\newtheorem{regle}{\r Règle}
\newtheorem{meth}{Méthode}
\newtheorem{dfn}{\r Définition}
\newtheorem{rmq}{Remarque}
\newtheorem{ex}{Exemple}
\newtheorem{exo}{Exercice}

\newcommand{\exemper}{\vskip 0.2cm\noindent{\noindent \bf \underline{Exemple \theexemple} : }\addtocounter{exemple}{1}}
\crefname{exemple}{Exemple}{exemples}

\newcommand{\methode}{\noindent \ul{\textbf{Méthode:}} }

\newenvironment{colordef}{\textbf{\color{myr}{Définition : }}\color{myb}}{}

%% Boite d'exemple

\newtcolorbox{exemple}{
	% breakable,
	colback=white,
	colframe=black,
	enhanced,
	attach boxed title to top left={xshift=5mm, yshift=-2mm},
	boxed title style={size=small, colback=white, colframe=black, arc=0.5mm},
	title=\textcolor{black}{Exemple \theexemple} \addtocounter{exemple}{1}, 
	arc=3mm,
}

% Macro pour les exercices et leur numérotation  
\newcounter{exemple}
\setcounter{exemple}{1}

\newcommand{\raisedrule}[2][0em]{\leaders\hbox{\rule[#1]{1pt}{#2}}\hfill}

\newcounter{countexo}
\setcounter{countexo}{1}
\newcommand{\exercice}{\vskip 0.2cm\noindent{ \bf \ \underline{Exercice \thecountexo :}}\addtocounter{countexo}{1}}

\newcounter{numexo}
\setcounter{numexo}{1}
\newcommand{\exerciceDS}[1]{\vskip 0.6cm\noindent\rule[+2pt]{1cm}{0.5pt}\ {\large\bf \ Exercice \thenumexo {  #1 }}\raisedrule[+2pt]{0.5pt}\null\addtocounter{numexo}{1}\par \vskip 0,4cm}

%%%%%%%%%%%%%%%%%%%
%-----Macros-----%
%%%%%%%%%%%%%%%%%%%


%% Couleurs
\definecolor{myb}{RGB}{10, 70, 93}
\definecolor{myg}{RGB}{0, 169, 51}
\definecolor{myr}{RGB}{218, 41, 26}
\definecolor{mybeige}{RGB}{255, 238, 208}
\definecolor{myrl}{RGB}{204, 102, 102}
\definecolor{mygris}{RGB}{230, 236, 238}
\definecolor{myjaune}{RGB}{255, 217, 102}
\definecolor{mygreen}{RGB}{121, 172, 120}
\renewcommand{\b}{\color{myb}}
\renewcommand{\r}{\color{myr}}
\newcommand{\ver}{\color{mygreen}}
\newcommand{\para}{\slash \slash}

\sethlcolor{myjaune}
\setul{0.5ex}{0.2ex}
\setulcolor{myr}

\newcommand\mover[1]{{\ver #1}}

%% Difficultés des exercices
\newcommand{\unet}{\ding{102} }
\newcommand{\deuxet}{\ding{102}\ding{102} }
\newcommand{\troiset}{\ding{102}\ding{102}\ding{102} }

%% Savoir-faire
\newcounter{numsf}
\setcounter{numsf}{1}
\newcommand\sfper[1]{\noindent {\color{black} \textbf{SF0\numchap.\thenumsf} - #1.} \addtocounter{numsf}{1}}


%% Raccourcis de commandes mathématiques
\newcommand{\R}{\mathbb{R}}
\newcommand{\Q}{\mathbb{Q}}
\newcommand{\D}{\mathbb{D}}
\newcommand{\Z}{\mathbb{Z}}
\newcommand{\N}{\mathbb{N}}
\newcommand{\Ca}{\mathcal{C}}

\newcommand{\lra}{\Leftrightarrow}
\newcommand{\ora}{\overrightarrow}

\newcommand\coord[2]{\begin{pmatrix}
 #1 \\
 #2
\end{pmatrix}}

\newcommand{\norme}[1]{\left\Vert #1\right\Vert}
\newcommand{\abs}[1]{\left\lvert#1\right\rvert}

\newcommand{\intervalleoo}[2]{\mathopen{]}#1\,;#2\mathclose{[}}
\newcommand{\intervalleff}[2]{\mathopen{[}#1\,;#2\mathclose{]}}
\newcommand{\intervalleof}[2]{\mathopen{]}#1\,;#2\mathclose{]}}
\newcommand{\intervallefo}[2]{\mathopen{[}#1\,;#2\mathclose{[}}
\newcommand{\intervalle}[2]{\mathopen{(}#1\,;#2\mathclose{)}}

%% Droites graduées
%\droite gradu\'ee{Nb de divisions de la longueur unit\'e}{longueur axe}
\newcommand{\droite}[2]{
	\psline{->}(0,0)(#2,0)
	\FPdiv{\division}{1}{#1}%Variable pour la division de la longueur Unit\'e.
	\FPmul{\produit}{#1}{#2}%Variable pour les graduations sur toute la longueur de l'axe.
	\FPclip{\produit}{\produit}%Supression des z\'eros inutiles.
	\multido{\r=0+\division}{\produit}%Boucle PsTricks
	{
		\psline(\r,-0.075)(\r,0.075)
	}
}

%\droite gradu\'ee avec nombres{Nb de divisions de la longueur unit\'e}{longueur axe}
\newcommand{\droitenombres}[2]{
	\psline{->}(0,0)(#2,0)
	\FPdiv{\division}{1}{#1}
	\FPmul{\produit}{#1}{#2}
	\FPclip{\produit}{\produit}
	\multido{\r=0+\division}{\produit}
	{
		\psline(\r,-0.075)(\r,0.075)
	}
	\multido{\i=0+1}{#2}{
		\psline[linewidth=0.03](\i,-0.15)(\i,0.15)
		\rput(\i,0.35){\i}}
}

%\droite gradu\'ee avec nombres{Nb de divisions de la longueur unit\'e}{longueur axe}{nb d\'epart}
\newcommand{\droitenombresorigine}[3]{
	\psline{->}(0,0)(#2,0)
	\FPdiv{\division}{1}{#1}
	\FPmul{\produit}{#1}{#2}
	\FPclip{\produitentier}{\produit}
	\multido{\r=0+\division}{\produitentier}
	{
		\psline(\r,-0.075)(\r,0.075)
	}
	
	\FPset{\nbchoisi}{#3}
	\multido{\i=0+1}{#2}{
		\psline[linewidth=0.03](\i,-0.15)(\i,0.15)
		\rput(\i,0.35){\nbchoisi}
		\FPadd{\nbchoisi}{1}{\nbchoisi}
		\FPclip{\nbchoisi}{\nbchoisi}
	}
}


%\droite gradu\'ee avec nombres{Nb de divisions de la longueur unit\'e}{longueur axe}{nb d\'epart}{pas}
\newcommand{\droitenombresoriginepas}[4]{
	\psline{->}(0,0)(#2,0)
	\FPdiv{\division}{1}{#1}
	\FPmul{\produit}{#1}{#2}
	\FPclip{\produitentier}{\produit}
	\multido{\r=0+\division}{\produitentier}
	{
		\psline(\r,-0.075)(\r,0.075)
	}
	\FPset{\nbchoisi}{#3}
	\multido{\i=0+1}{#2}{
		\psline[linewidth=0.03](\i,-0.15)(\i,0.15)
		\rput(\i,0.35){\nbchoisi}
		\FPadd{\nbchoisi}{#4}{\nbchoisi}
		\FPclip{\nbchoisi}{\nbchoisi}
	}
}